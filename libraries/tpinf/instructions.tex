\documentclass[a4paper,10pt]{article}

\usepackage[utf8]{inputenc}
\usepackage[margin=2cm]{geometry}
\usepackage[colorlinks=true, linkcolor=blue, anchorcolor=blue,
citecolor=blue, filecolor=blue, menucolor=blue,
urlcolor=blue]{hyperref}

\setlength{\parskip}{3mm}
\setlength{\parindent}{0mm}

\begin{document}
\centerline{\Large \sc Instructions for the test}
\vspace{5mm}
\hrule
\vspace{5mm}
\tableofcontents

\section{Getting started}

We consider in the followwing that you have opened a terminal, and that you are placed in the directory where the exam materials lies.

In that terminal, typing only
\begin{verbatim}make
\end{verbatim}
helps you. It displays help. Read the forewords displayed by this command.

\section{How to pass the test}

The exam is organized into parts. In each part, there are several questions, that have to be considered in order.

For part~1 (the same stands for part 2, 3, ...) you are given the following {\tt C++} files. {\bf Just read the following}, you will have to really do something at section~\ref{sec:starting}.
\begin{itemize}
  \item {\tt part1.hpp}: This is where you will have to write code when you will be asked to do so.
  \item {\tt part1.cpp}: May not be present. If it exists, you will be asked to write code in it as well.
  \item {\tt question1-part1.cpp}: This file contains the instructions corresponding to first question of part~1. {\bf Do not modify this file}\footnote{Except if you are explicitly told to do so.}. Read it, comments will tell you what to do, where to write code, etc. If you do things right, compiling {\tt question1-part1.cpp} should succeed and it will output what is expected.
  \item {\tt question2-part1.cpp}: Next question...
  \item {\tt question3-part1.cpp}: Next question...
  \item {\tt ...}
\end{itemize}

Once you have implemented what is required by, let us say, {\tt question3-part2.cpp}, you have to compile the {\tt question3-part2.cpp} and run the resulting program. {\bf Do not invoke g++ directly}, we provide you with this simplified command.
\begin{verbatim}make question3-part2
\end{verbatim}

You will not have to type everything if you use the completion key (the {\tt TAB} key).

In case of compiling errors, output that not fit the requirements, you will have to modify your code (in {\tt part2.hpp} or {\tt part2.cpp} in this example), and retry the command until everything succeeds. 

\section{Start the test \label{sec:starting}}

Now, type

\begin{verbatim}make
\end{verbatim}

to display all the questions (lines like {\tt make question-X-part-Y}). Edit the corresponding {\tt c++} file with one of the available code editors. This file is {\tt question-X-part-Y.cpp} if you have typed {\tt make question-X-part-Y}. Reading the comments tells you what to do (don't modify that file if you are not explicitly asked to). Compile and test that question by typing the corresponding 

\begin{verbatim}make questionX-partY
\end{verbatim}

command.

\section{Warnings}

The documentation is available on this machine, you have no access to internet, and no extra electronic devices are allowed.

{\bf DO NOT} access collections elements with the {\tt []} operator, like in {\tt  tab[4]}, since this is not efficient within loops.

Each function you will have to implement {\bf is short} (less than 10 lines). Do not get lost in obfuscated code !

Be sure to save the files you have modified before the end of the test. Do not create new files, anwsering consists in modifying the given files.

Use {\tt make oops} in case of accidental file deletions... but the best is to avoid it.



\end{document}
